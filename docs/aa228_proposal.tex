\documentclass{article}
\usepackage[utf8]{inputenc}
% \usepackage{indentfirst}
\usepackage{hyperref}
\usepackage[a4paper, total={6in, 10in}]{geometry}


\title{% 
    AA228 Proposal: Formal Proofs or Metabolic Pathways}
\author{Daniel Jenson, Daniel Huang and Julian Cooper}
% \date{January 20th, 2023}

\begin{document}

\maketitle

\textbf{Context}: Our team is made up of students in both CS 238 (AA228) and
CS224W (graph neural networks). We are hoping to do a combined project where we
tackle a sequential decision-making problem and use graph neural network
techniques to choose the "next besit action" at each step. We have been
brainstorming a lot this week (including with Prof. Kochenderfer) and have two
ideas (from which we'll pick one) we wanted to get feedback on: (1) formal
proof solving algorithm and (2) metabolic pathway finding algorithm. 

\section{Idea 1: Formal proof solving algorithm}
\subsection{Overview}
There were several papers released in 2019 that attempted to use the power of
neural networks in conjunction with automated theorem provers (ATP) like HOL
light and Coq. For this idea, we chose to focus on Coq, since this
\href{https://arxiv.org/abs/1905.09381}{paper} provides an environment
\texttt{CoqGym} as well as a reference implementation of their
\texttt{TreeLSTM}. In the \texttt{CoqGym} environment, Yang and Deng found they
could prove 30\% of theorems using a combination of their \texttt{TreeLSTM} and
hammer (a collection of 3rd party ATPs) compared to 4.4\% using the built-in
ATP.  Paliwal et al. found in \href{https://arxiv.org/abs/1905.10006}{paper}
that they could prove up to 50\% of theorems in the HOL List dataset with a
12-hop GNN and could outperform most existing methods with a simple
bag-of-nodes based approach.
\subsection{Algorithm and dataset}
\textbf{Algorithm}: The general idea is that there is a feedback loop between
the ATP Coq and an agent powered by a neural network. We would like to train
the agent to select better premises that can be used as proof tactics in
proving a goal or sub-goal. In a sense, we are training an agent to learn to
generalize from proof structures so it can sample higher value premises or
tactics. We hope that by using recent advances in graph neural networks, we can
create a more robust embedding using hierarchical graph neural networks (GNNs)
as well as (possibly) gated updates (using GRUs or LSTMs) and graph attention
mechanisms (GATs). Most of the papers released on this topic are from 2019,
which is when PyTorch Geometric (PyG) was just starting to be developed.

\textbf{Dataset}: The dataset we would use is embedded in the environment
\texttt{CoqGym} and consists of 70,856 human-written proofs. For training
purposes, the authors also extracted approximately 160,000 one-step proofs,
110,000 two-step proofs, and 80,000 three-step proofs, and 61,000 four-step
proofs. For reference, the average number of steps per complete proof was 9.1.

\subsection{Risks \& open questions}
\begin{enumerate}
    \item \textbf{Domain knowledge}: Not everyone in the group has a background
      in theoretical math or theorem proving. No one in the group has used Coq
      before.
    \item \textbf{Training time}: If we set the proof timeout at 20 seconds, it would take approximately 16 days to run over the entire dataset of 71,000 proofs. Accordingly, we will need to select a subset of proofs to iterate on before training the full model. However, we believe that we can train the graph neural network representations independently of the agent that interacts with the ATP, since these embeddings are based on graph structure.
\end{enumerate}


\section{Idea 2: Metabolic pathway finding algorithm}
\subsection{Algorithm and dataset}
\textbf{Algorithm}: Provided a target molecule and a set of potential start molecules (abundantly available compounds), we want out algorithm to find the shortest feasible reaction pathway. To structure this as a sequential decision-making problem, the model (agent) proposes a reaction (action) to progress the state of the molecule towards the target(s), iterating until this until either target is reached (reward = 1) or not (reward = some closeness metric to nearest goal). Most metabolic pathways take multiple reaction steps to go from start to target with multiple branching dependencies, which makes the number of potential reactions available to complete a path too large to solve deterministically. Rather we will utilize a neural network agent observing embeddings of molecule state and available reactions (from the graph structure) to select an action that will lead us closer to our target.\\

\textbf{Dataset}: The Kyoto Encyclopedia of Genes and Genomes (KEGG) is an industry-leading database of metabolic pathways that can be used for training and testing. Moreover, there is a number of well-maintained software packages that are built around the KEGG API. In particular, we would propose the following pipeline: PathMe  or Bio2Bel to convert KEGG data into BEL (Biological Expression Language), then PyBel to convert BEL into usable format for graph packages like \texttt{networkx}, and finally PyKEEN which takes PyBel format and supports graph neural network training and evaluation.

\subsection{Risks \& open questions}
\begin{enumerate}
    \item \textbf{Reaction simulator}: The main concern is whether we need to build a simulator from scratch that takes a molecule state and a reaction and produces a new model state. Ideally something like this already exists and is open source so we can focus on the algorithm itself for this project. We have been able to identify a range of academic papers that use the software pipeline described above for "classification" problems (eg. given a graph embedding of a pathway, predict the class of pathway), but nothing so far that treats the molecule at intermediate steps in the pathway as a state to act on iteratively (would require a simulator of some sort). 
    
    \item \textbf{Topic expertise}: While this is an area of interest for our team, none of us have formal training in biology or chemistry beyond first year undergraduate (organic chem). This may make it difficult to qualitatively assess whether or model is making "good" decisions and come up with design choices that take advantage of chemical reaction laws (equivalent to encoding conservation of mass for mechanical engineering projects).
\end{enumerate}

\end{document}
